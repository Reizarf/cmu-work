% !TEX TS-program = pdflatex
% !TEX encoding = UTF-8 Unicode

% This is a simple template for a LaTeX document using the "article" class.
% See "book", "report", "letter" for other types of document.

\documentclass[12pt]{article} % use larger type; default would be 10pt

\usepackage[utf8]{inputenc} % set input encoding (not needed with XeLaTeX)

%%% Examples of Article customizations
% These packages are optional, depending whether you want the features they provide.
% See the LaTeX Companion or other references for full information.

%%% PAGE DIMENSIONS
\usepackage{geometry} % to change the page dimensions
\geometry{a4paper} % or letterpaper (US) or a5paper or....
% \geometry{margin=2in} % for example, change the margins to 2 inches all round
% \geometry{landscape} % set up the page for landscape
%   read geometry.pdf for detailed page layout information
\usepackage{tabto}
\usepackage{graphicx} % support the \includegraphics command and options

% \usepackage[parfill]{parskip} % Activate to begin paragraphs with an empty line rather than an indent

%%% PACKAGES
\usepackage{booktabs} % for much better looking tables
\usepackage{array} % for better arrays (eg matrices) in maths
\usepackage{paralist} % very flexible & customisable lists (eg. enumerate/itemize, etc.)
\usepackage{verbatim} % adds environment for commenting out blocks of text & for better verbatim
\usepackage{subfig} % make it possible to include more than one captioned figure/table in a single float
% These packages are all incorporated in the memoir class to one degree or another...

%%% HEADERS & FOOTERS
\usepackage{fancyhdr} % This should be set AFTER setting up the page geometry
\pagestyle{fancy} % options: empty , plain , fancy
\renewcommand{\headrulewidth}{0pt} % customise the layout...
\lhead{}\chead{}\rhead{}
\lfoot{}\cfoot{\thepage}\rfoot{}

%%% SECTION TITLE APPEARANCE
\usepackage{sectsty}
\allsectionsfont{\sffamily\mdseries\upshape} % (See the fntguide.pdf for font help)
% (This matches ConTeXt defaults)

%%% ToC (table of contents) APPEARANCE
\usepackage[nottoc,notlof,notlot]{tocbibind} % Put the bibliography in the ToC
\usepackage[titles,subfigure]{tocloft} % Alter the style of the Table of Contents
\renewcommand{\cftsecfont}{\rmfamily\mdseries\upshape}
\renewcommand{\cftsecpagefont}{\rmfamily\mdseries\upshape} % No bold!

%%% END Article customizations

%%% The "real" document content comes below...

\title{Academic Integrity Essay}
\author{Sullivan Frazier}
%\date{} % Activate to display a given date or no date (if empty),
         % otherwise the current date is printed 

\begin{document}
\maketitle

\section{If you were in charge of CMU, what four initiatives would you start to promote and ensure academic integrity on campus?}

\tab

	In order to truly put up a fight against plagiarism, there’s a few ideas I believe we can work with. I believe we need to start working in a couple different areas, from on campus to in the classrooms. 

I see a potential opportunity to initiate a strength-y tutoring system, hopefully putting an end to the “need” to look at someone else’s work from the foundation.  There is not a need for you to look around, when you completely understand the topic in the first place. Although this is obviously not bullet proof, it is giving a reasonable band aid, and helping stop the spread of plagiarism from the source of the flames. Although the TLC is a valid representation of that, I think we could dive deeper with individual majors - like what has happened in the past years with CS tutors! This was extremely helpful as the tutors had direct experience with the classes, professors, and had a vague memory of how the assignment was supposed to go. Having someone that’s already been through it was extremely helpful in my eyes, as they know exactly how it feels to be learning this stuff. They were students too, and they’ve been in my exact academic position before.

Another initiative has to deal with an anonymous reporting system in place. I’m not sure if this violates guidelines of any kind, but it’s definitely an idea. Maybe we can implement an anonymous cheating reporting system - hopefully enforcing integrity to begin with. If your peers are on the lookout for cheating, the students will want to pay attention and do well in class, keeping integrity on the top of their list. As well as keeping an eye on others for any unauthorized collaboration -  so there’s never any need to lean on anyone else. This sort of rolls into my next idea - propaganda. 

Have you ever been in the bathrooms or around on campus and seen posters for events and such? What if we can develop some sort of propaganda if you will, that’s “friendly” while still warning against plagiarism and cheating. These attempts can be conveyed as malicious, or typically unacceptable - the idea and constant persuading within propaganda may have a beneficial effect in the long run. 

My last initiative sort of rolls back into the first one, but more on the defensive side. The promotion of a “citing sources/education campaign tutoring center” could be an idea. If students were beaten into learning citing sources, I think the plagiarism rate would decline - maybe not drastically but it could most definitely help. I realize some of my peers are still searching on how to correctly cite - which can get frustrating and sometimes ends up being cited incorrectly, or even worse, not at all. Then come the consequences on a student that is academically lost. 


\section{Who is your role model? Discuss, citing specific instances, how that role model lived an ethical life and what traits he or she possessed which you admire most. }

\tab
As simple and go to my answer may seem, my role model is and always will be my father. Of course he is intelligent, but he has an ability to evaluate a situation with many things considered - and just about most of the time it works out in our favor. Taking elements into consideration like financial situation, ethical decisions and matters, attitude and approach, and many more. Of course he is trustworthy, responsible, and very good at time management, but most of all his integrity has always been in the correct field. He was never malicious with his actions or words, in any manner.  I admire his integrity the most - it’s not him being selfless, but not selfish either. There’s a balance and I am striving to find that tipping point and maintain that balance. He’s been in the field for years now, selling cyber security products to multi-million dollar enterprises. Years back he even had a security clearance, that was definitely an experience growing up - he was traveling a TON. Nevertheless he’s dealt with a few high-stress situations, and this has definitely formed him into the character he is today.

\section{Expand on the following statement and apply it to yourself: “A student’s adherence to academic integrity is a reflection on their personal values.” and enumerate academic integrity violation sanctions (all) as outlined in your course syllabus.}


\tab Academic Integrity is of course a huge element to a student’s success and opportunities, but I think it applies more to general humanitarian integrity. If your intentions are malicious, odds are you are not/will not become successful in the long run, and that applies to more than academic scenarios. Integrity is something that needs to be applied and followed through on a personal level, before the student ever enters an academic environment. 

Prevention against forgery, falsification and plagiarism of academic documents.

Ex. Thievery of code that’s online - while still claiming you wrote it or it’s yours. Using other sources and copying/pasting. 

Intentionally impeding the academic work of others.

Ex. Looking on their screen, their papers, or code without authorized permission. Asking for a copy or something of that nature.

Aiding other students in academic dishonesty

Ex. Stealing other students' information and spreading the criteria is aiding academic dishonesty. Spreading criteria that’s been stolen in the first place, is aiding dishonesty. 

Cheating directly in the classroom

Ex. Tests, quizzes, etc being taken locally, it’s a cheating playground for some students when the test is being taken locally. Potentially even more so if it’s online.

Unauthorized attendance (?)

Ex. I’m not 100 percent sure what this means, other than sitting in on a class you’re not supposed to be in? Maybe sneaking into a meeting, or a higher up class or something of that nature?

Multiple Submissions

Ex. Submitting your final revision several times over, some professors allow this while others do not. 
Last but not least, Unauthorized collaboration. 

Ex. Students are collaborating on something that is supposed to be completed independently. 

Unauthorized use of materials or equipment to complete an academic requirement.

Ex. I believe this refers to using external technology or something of that nature, in order to complete an academic assignment.


\section{Consider the “Learning Outcomes” as cited in the Colorado Mesa University Strategic Plan. What does the University hope to accomplish by promoting these outcomes?}

\tab 
The Learning outcomes are clearly labelled topics/subjects that the student is supposed to be able to perform and/or elaborate on, by the time the class comes to an end. The University is trying to set up these learning ‘targets’, I sort of see them as goals, and throughout the semester the student is working to master these ‘targets' or goals. Sometimes they also point towards the general content that will be covered in that specific class throughout the semester. They’re labeled as student learning outcomes, (SLO’s) and are tacked onto a lot of the syllabi here at CMU as well. By promoting these outcomes, it sets an end goal for the professor AND student, while being approved by CMU all simultaneously and providing an end goal. 


\section{Why is citing sources an important part of academic work? Discuss some real-world implications of not properly citing sources}
\tab
You mentioned something specific in class on the first day back that stuck with me. If you’re working with something specific, for example the shell sort algorithm,  you are discussing bits of information, describing the algorithm itself and how it works - you must still cite from the original author/creator, Donald Shell. It is not new information you are creating/publishing, nor are you relaying anything new. The truth is you are conveying information that’s not yours, but it’s been digested mentally and utilized in your education - therefore the credit must be given and it must be cited in your work to the rightful creator and/or owner. You are not the creator, nor are you doing something new - therefore credit must be given to the rightful creator or owner. Same thing applies with a lot of other scenarios, other sorting algorithms, and many other coding instances. 


\section{How does not reporting an academic integrity rules violation reflect on one’s academic integrity? Provide some scenarios when it is tempting to not report a violation and why it is better in the long run to use integrity and responsibility in such situations.}
\tab
The work can potentially become traced back to you - meaning you released your own work to other students willingly. That is just as punishable as the student that cheated, meaning you have false integrity. Maintaining integrity and responsibility will never ever become detrimental to you or your success in academic environments. 


\section{ Explore the favorable and adverse effects of having someone help you on an assignment or test. How does it advantage each of you, and how does it disadvantage each of you? Develop a final statement and plan of what you will do in the future if you are not properly prepared for a task or if someone asks you for aid.}
\tab
Sometimes a peer to peer aid on an assignment can be beneficial. The students learn collaboration and how to discuss ideas with each other, while debating on a final route to execute. Having just another pair of eyes can be helpful to another student with a minimal mistake, while on the other hand with collaboration being present, sometimes a student's growth can be hindered. Collaboration can in a way prevent the student from experiencing other learning methods, such as trial and error, pure exposure, and/or self-teaching. If I am asked for aid with a task, I’d refer them to where they could get that information, but not just plain give it out - that’s the hunt, right? Maybe a website link, or a reference to a chapter in the course's textbook, or something along those lines. There’s no need to give a direct answer, what if you just gently guide them in the right direction? This is just as helpful, and oftentimes is better for the student. It teaches them how to find their answers, while also not being left in the complete abyss. 


\section{Homework is a necessary and critical component for learning and understanding the material in a difficult class. Discuss some ideas as to why a professor would not want collaboration on homework.}
\tab
Peer Collaboration can potentially have detrimental effects to the students’ academic growth. Although the intentions can sometimes be harmless, sometimes the natural mental progression of a student is what is truly beneficial academically - especially when they’re independent of each other. A student being able to go from thought chain to another, and potentially connecting the dots and leading to a valuable solution is the epitome of self teaching. This is how a student flourishes. Homework is a critical element to education due to the fact that the student is being exposed to the content, out of their normally routined class schedule - and this is what typically makes it as effective as it is. Oftentimes it is difficult, enforcing the student to focus and spend a decent amount of time towards perfecting their assignment. This is why it is beneficial, as it is an extra dose of exposure to the young mind. 



%\subsection{I declare that all material in this assessment task is my work except where there is clear acknowledgement or reference to the work of others. I further declare that I have complied and agreed to the CMU Academic Integrity Policy at the University website.
%http://www.coloradomesa.edu/student-services/documents
%Author’sName: .......Sullivan Frazier........ UID(700): ..700-479463..... Date: .....8/28/21…
%Submissions that do not include the above academic integrity statement will not be considered.}



\end{document}
